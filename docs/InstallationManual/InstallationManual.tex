\documentclass{article}
\usepackage[a4paper]{geometry}
\usepackage[utf8]{inputenc}
\usepackage{polski}
\usepackage{tabularx}
\usepackage{indentfirst}
\usepackage{multirow}
\usepackage{amssymb}
\usepackage{amsmath}
\usepackage{anysize}
\usepackage{float}
\usepackage{caption}
\usepackage{subcaption}
\usepackage{graphicx}

\usepackage{listings}
\usepackage{color}
\lstset{literate=%
{ą}{{\k{a}}}1 {ć}{{\'c}}1 {ę}{{\k{e}}}1 {ł}{{\l{}}}1 {ń}{{\'n}}1 {ó}{{\'o}}1 {ś}{{\'s}}1 {ż}{{\.z}}1 {ź}{{\'z}}1 {Ą}{{\k{A}}}1 {Ć}{{\'C}}1 {Ę}{{\k{E}}}1 {Ł}{{\L{}}}1 {Ń}{{\'N}}1 {Ó}{{\'O}}1 {Ś}{{\'S}}1 {Ż}{{\.Z}}1 {Ź}{{\'Z}}1 }

\definecolor{mygreen}{rgb}{0,0.6,0}
\definecolor{mygray}{rgb}{0.5,0.5,0.5}
\definecolor{mymauve}{rgb}{0.58,0,0.82}

\usepackage{titling}
\newcommand{\subtitle}[1]{%
	\posttitle{%
	\par\end{center}
	\begin{center}\small#1\end{center}
	\vskip0.5em}%
}

\title{Algenic -- serwis z~konkursami algorytmicznymi\\
Instrukcja instalacji, konfiguracji}
\subtitle{Akademia Górniczo-Hutnicza im. Stanisława Staszica w~Krakowie\\
	Wydział Elektrotechniki, Automatyki,\\
	Informatyki i Inżynierii Biomedycznej}
\author{Kacper Tonia\and
		Sławomir Kalandyk\and
		Mateusz~Ruciński}
\date{}

\begin{document}
%------------------------------------------------------------
\maketitle

\section{Korzystanie z~aplikacji}
\subsection{Instalacja}
Wymagania:
\begin{itemize}
    \item .NET Core SDK 2.2 lub nowszy
    \item Visual Studio~2017 lub nowszy (opcjonalnie)
\end{itemize}

\subsection{Uruchomienie}
Aby uruchomić aplikację, można skorzystać z~konsolowego~narzędzia dotnet. Należy w~tym celu:
\begin{enumerate}
    \item Przejść do~katalogu, w~którym znajduje się projekt.
    \item Wejść do~katalogu Algenic/Algenic.
    \item Wykonać polecenie "dotnet run".
\end{enumerate}

\subsection{Testy jednostkowe}
\begin{enumerate}
    \item Upewnić się, że~aplikacja nie jest uruchomiona np.~poprzez~dotnet.
    \item Uruchomić Visual Studio.
    \item Z~poziomu okna Test Explorer wybrać Algenic.UnitTests $\rightarrow$ Run.
\end{enumerate}

\subsection{Testy funkcjonalne}
Specyfika uruchamiania testów~funkcjonalnych w~projekcie sprawia, że~nie mogą być one uruchamiane razem z~testami jednostkowymi. Ponadto, konieczna jest już działająca instancja aplikacji Algenic w~momencie uruchamiania testów.

\begin{enumerate}
    \item Upewnić się, że~co~najmniej jedna ze wspieranych przeglądarek (Firefox, Chrome, Opera) jest zainstalowana. Wybrać odpowiednią z~nich w~pliku konfiguracyjnym\\Algenic/Algenic.FunctionalTests.testSettings.json.
    \item Dla wybranej przeglądarki pobrać WebDriver, który pozwoli frameworkowi testowemu komunikować się z~przeglądarką. Pobrany WebDriver (np.~operadriver.exe) umieścić w~katalogu Algenic/Algenic.FunctionalTests/bin/Debug/netcore2.2.
    \item Uruchomić aplikację z~poziomu narzędzia dotnet (dotnet run).
    \item W~Visual Studio, w~oknie Test Explorer, wybrać Algenic.FunctionalTests $\rightarrow$ Run.
\end{enumerate}
\end{document}
