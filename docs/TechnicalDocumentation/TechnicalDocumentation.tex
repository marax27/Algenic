\documentclass{article}
\usepackage[a4paper]{geometry}
\usepackage[utf8]{inputenc}
\usepackage{polski}
\usepackage{tabularx}
\usepackage{indentfirst}
\usepackage{multirow}
\usepackage{amssymb}
\usepackage{amsmath}
\usepackage{anysize}
\usepackage{float}
\usepackage{caption}
\usepackage{subcaption}
\usepackage{graphicx}

\usepackage{listings}
\usepackage{color}
\lstset{literate=%
{ą}{{\k{a}}}1 {ć}{{\'c}}1 {ę}{{\k{e}}}1 {ł}{{\l{}}}1 {ń}{{\'n}}1 {ó}{{\'o}}1 {ś}{{\'s}}1 {ż}{{\.z}}1 {ź}{{\'z}}1 {Ą}{{\k{A}}}1 {Ć}{{\'C}}1 {Ę}{{\k{E}}}1 {Ł}{{\L{}}}1 {Ń}{{\'N}}1 {Ó}{{\'O}}1 {Ś}{{\'S}}1 {Ż}{{\.Z}}1 {Ź}{{\'Z}}1 }

\definecolor{mygreen}{rgb}{0,0.6,0}
\definecolor{mygray}{rgb}{0.5,0.5,0.5}
\definecolor{mymauve}{rgb}{0.58,0,0.82}

\usepackage{titling}
\newcommand{\subtitle}[1]{%
	\posttitle{%
	\par\end{center}
	\begin{center}\small#1\end{center}
	\vskip0.5em}%
}

\title{Algenic -- serwis z~konkursami algorytmicznymi\\
    Dokumentacja techniczna}
\subtitle{Akademia Górniczo-Hutnicza im. Stanisława Staszica w Krakowie\\
	Wydział Elektrotechniki, Automatyki,\\
	Informatyki i Inżynierii Biomedycznej}
\author{Kacper Tonia\and
		Sławomir Kalandyk\and
		Mateusz Ruciński}
\date{}

\begin{document}
%------------------------------------------------------------
\maketitle

\section{Struktura projektu}
Algenic składa się z następujących projektów:
\begin{itemize}
    \item \textit{Algenic}: pliki aplikacji sieciowej (kod HTML, CSS itd.), logika aplikacji.
    \item \textit{Algenic.Commons}: podstawowe interfejsy, narzędzia niebędące związane ze specyfiką aplikacji.
    \item \textit{Algenic.Compilation}: stanowi warstwę abstrakcji w procesie kompilacji kodu. Zawiera przede wszystkim klasy pozwalające na komunikację z API serwisu JDoodle.
    \item \textit{Algenic.UnitTests}: testy jednostkowe.
    \item \textit{Algenic.FunctionalTests}: testy Selenium.
\end{itemize}

\section{Struktura strony internetowej}
\begin{itemize}
    \item Strona główna: zalogowany użytkownik może tutaj obejrzeć wyniki konkursów, w których brał udział.
    \item \emph{/ScorePolicies}: służy tworzeniu nowych polityk oceniania. Dostęp tylko dla egzaminatorów.
    \item \emph{/Contests}: umożliwia przeglądanie, dołączanie do istniejących konkursów. Egzaminatorzy mogą dodatkowo tworzyć nowe konkursy bądź edytować istniejące.
    \item \emph{/Admin}: panel administratora
\end{itemize}

\section{Testy automatyczne}
Testy automatyczne w projekcie dzielą się na 2 grupy: jednostkowe i funkcjonalne.
\subsection{Testy jednostkowe}
Aby ułatwić testowanie, operacje na bazie danych zostały wydzielone i zawarte w tzw. handlerach. QueryHandler ma za zadanie pobrać dane z bazy i zwrócić wynik, natomaist CommandHandler dodaje nowe dane bądź modyfikuje istniejące wpisy. Dla każdego testu tworzona jest w pamięci operacyjnej (in-memory) tymczasowa baza danych, zatem testy nie wpływają ani na siebie, ani na oryginalną bazę.
\subsection{Testy funkcjonalne}
Testy funkcjonalne wykorzystują 3 konta użytkowników, tworzonych domyślnie, gdy aplikacja jest uruchamiana w trybie developerskim.

Obecne testy funkcjonalne badają następujące funkcjonalności programu:
\begin{enumerate}
    \item Dostęp do panelu administratora
    \begin{itemize}
        \item jako zwykły użytkownik (brak dostępu)
        \item jako egzaminator (brak dostępu)
        \item jako administrator (dostęp przyznany)
    \end{itemize}
    \item Dodawanie konkursów
    \begin{itemize}
        \item egzaminator tworzy nowy konkurs
        \item dane nowego konkursu powinny pojawić się w tabeli konkursów
    \end{itemize}
    \item Uprawnienia do dodawania konkursów
    \begin{itemize}
        \item jako egzaminator (pojawia się odpowiedni formularz)
        \item jako zwykły użytkownik (brak formularza na stronie)
    \end{itemize}
    \item Zmiana właściciela konkursu przy odebraniu uprawnień egzaminatora obecnemu właścicielowi konkursu
\end{enumerate}

\end{document}