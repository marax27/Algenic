\documentclass{article}
\usepackage[a4paper]{geometry}
\usepackage[utf8]{inputenc}
\usepackage{polski}
\usepackage{tabularx}
\usepackage{indentfirst}
\usepackage{multirow}
\usepackage{amssymb}
\usepackage{amsmath}
\usepackage{anysize}
\usepackage{float}
\usepackage{caption}
\usepackage{subcaption}
\usepackage{graphicx}

\usepackage{listings}
\usepackage{color}
\lstset{literate=%
{ą}{{\k{a}}}1 {ć}{{\'c}}1 {ę}{{\k{e}}}1 {ł}{{\l{}}}1 {ń}{{\'n}}1 {ó}{{\'o}}1 {ś}{{\'s}}1 {ż}{{\.z}}1 {ź}{{\'z}}1 {Ą}{{\k{A}}}1 {Ć}{{\'C}}1 {Ę}{{\k{E}}}1 {Ł}{{\L{}}}1 {Ń}{{\'N}}1 {Ó}{{\'O}}1 {Ś}{{\'S}}1 {Ż}{{\.Z}}1 {Ź}{{\'Z}}1 }

\definecolor{mygreen}{rgb}{0,0.6,0}
\definecolor{mygray}{rgb}{0.5,0.5,0.5}
\definecolor{mymauve}{rgb}{0.58,0,0.82}

\usepackage{titling}
\newcommand{\subtitle}[1]{%
	\posttitle{%
	\par\end{center}
	\begin{center}\small#1\end{center}
	\vskip0.5em}%
}

\title{Algenic -- serwis z~konkursami algorytmicznymi\\
Instrukcja użytkowania}
\subtitle{Akademia Górniczo-Hutnicza im. Stanisława Staszica w Krakowie\\
	Wydział Elektrotechniki, Automatyki,\\
	Informatyki i Inżynierii Biomedycznej}
\author{Kacper Tonia\and
		Sławomir Kalandyk\and
		Mateusz Ruciński}
\date{}

\begin{document}

\maketitle

\section{Role}
\subsection{Administrator}
Podstawową czynnością administracyjną jest przydzielanie uprawnień Egzaminatora. Jest to możliwe z poziomu Panelu Administratora dostępnego przez pasek menu u góry strony (dostęp tylko dla zalogowanego Administratora). Aby przydzielić/odebrać użytkownikowi prawa Egzaminatora, należy nacisnąć odpowiedni przycisk obok nazwy użytkownika.

Uwaga: odbierając uprawnienia Egzaminatora wszystkie jego konkursy zmienią właściciela: nowym właścicielem zostanie Administrator. Odpowiednie ostrzeżenie wyświetli się po najechaniu na przycisk "Revoke".

\subsection{Egzaminator}
Egzaminator może:
\begin{itemize}
    \item tworzyć konkursy
    \item usuwać swoje konkursy
    \item zmieniać stan swojego konkursu (Not started, In progress, Completed)
    \item edytować swoje (nierozpoczęte) konkursy, w tym dodawać, edytować i usuwać zadania
\end{itemize}

\subsection{Każdy użytkownik}
Każdy użytkownik może:
\begin{itemize}
    \item wysyłać rozwiązania do zadań z cudzych konkursów
    \item przeglądać wyniki swoich rozwiązań (po zakończeniu konkursu)
\end{itemize}

\section{Tworzenie konkursu}
\begin{enumerate}
\end{enumerate}


\end{document}