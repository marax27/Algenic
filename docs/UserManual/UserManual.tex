\documentclass{article}
\usepackage[a4paper]{geometry}
\usepackage[utf8]{inputenc}
\usepackage{polski}
\usepackage{tabularx}
\usepackage{indentfirst}
\usepackage{multirow}
\usepackage{amssymb}
\usepackage{amsmath}
\usepackage{anysize}
\usepackage{float}
\usepackage{caption}
\usepackage{subcaption}
\usepackage{graphicx}

\usepackage{listings}
\usepackage{color}
\lstset{literate=%
{ą}{{\k{a}}}1 {ć}{{\'c}}1 {ę}{{\k{e}}}1 {ł}{{\l{}}}1 {ń}{{\'n}}1 {ó}{{\'o}}1 {ś}{{\'s}}1 {ż}{{\.z}}1 {ź}{{\'z}}1 {Ą}{{\k{A}}}1 {Ć}{{\'C}}1 {Ę}{{\k{E}}}1 {Ł}{{\L{}}}1 {Ń}{{\'N}}1 {Ó}{{\'O}}1 {Ś}{{\'S}}1 {Ż}{{\.Z}}1 {Ź}{{\'Z}}1 }

\definecolor{mygreen}{rgb}{0,0.6,0}
\definecolor{mygray}{rgb}{0.5,0.5,0.5}
\definecolor{mymauve}{rgb}{0.58,0,0.82}

\usepackage{titling}
\newcommand{\subtitle}[1]{%
	\posttitle{%
	\par\end{center}
	\begin{center}\small#1\end{center}
	\vskip0.5em}%
}

\title{Algenic -- serwis z~konkursami algorytmicznymi\\
Instrukcja użytkowania}
\subtitle{Akademia Górniczo-Hutnicza im. Stanisława Staszica w Krakowie\\
	Wydział Elektrotechniki, Automatyki,\\
	Informatyki i Inżynierii Biomedycznej}
\author{Kacper Tonia\and
		Sławomir Kalandyk\and
		Mateusz Ruciński}
\date{}

\begin{document}

\maketitle

\section{Słowniczek}
\begin{itemize}
	\item Konkurs (Contest) - konkurs utworzony przez użytkownika z uprawnieniami egzaminatora. 
	\item Zadanie (Task)- zadanie dodane w ramach konkretnego konkursu, użytkownicy mogą przesyłać jego rozwiązania.
	\item Test (Test) - stanowi parę (wejście, spodziewane wyjście), jest przypisany do konkretnego zadania.
	\item Polityka punktowania (Score policy) - zbiór zasad punktowania. Wyznacza ile punktów w ramach zadania zdobywa użytkownik zależnie od ilości zaliczonych testów.
	\item Zasada punktowania (Score rule) - stanowi parę (próg, ilość punktów). Oznacza to, jaki próg musi przekroczyć użytkownik, aby zdobyć daną liczbę punktów za rozwiązane zadanie.
\end{itemize}

\section{Role}
\subsection{Administrator}
Podstawową czynnością administracyjną jest przydzielanie uprawnień Egzaminatora. Jest to możliwe z poziomu Panelu Administratora dostępnego przez pasek menu u góry strony (dostęp tylko dla zalogowanego Administratora). Aby przydzielić/odebrać użytkownikowi prawa Egzaminatora, należy nacisnąć odpowiedni przycisk obok nazwy użytkownika.

Uwaga: odbierając uprawnienia Egzaminatora wszystkie jego konkursy zmienią właściciela: nowym właścicielem zostanie Administrator. Odpowiednie ostrzeżenie wyświetli się po najechaniu na przycisk "Revoke".

\subsection{Egzaminator}
Egzaminator może:
\begin{itemize}
    \item tworzyć konkursy
    \item usuwać swoje konkursy
    \item zmieniać stan swojego konkursu (Not started, In progress, Completed)
    \item edytować swoje (nierozpoczęte) konkursy, w tym dodawać, edytować i usuwać zadania
\end{itemize}

\subsection{Każdy użytkownik}
Każdy użytkownik może:
\begin{itemize}
    \item wysyłać rozwiązania do zadań z cudzych konkursów
    \item przeglądać wyniki swoich rozwiązań (po zakończeniu konkursu)
\end{itemize}

\section{Tworzenie konkursu}
Wymagania:
\begin{itemize}
	\item uprawnienia egzaminatora
\end{itemize}
Proces tworzenia konkursu:
\begin{enumerate}
	\item Wejdź w zakładkę Contests na pasku nawigacji.
	\item Na szczycie strony powinno być dostępne pole tekstowe, w którym możesz wpisać nazwę nowego konkurs.
	\item Po wpisaniu nazwy naciśnij przycisk "Send". Nowy konkurs pojawia się na liście konkursów, możesz go edytować - zmienić nazwę, dodawać zadania, rozpocząć go.
\end{enumerate}

\section{Edycja istniejącego konkursu}
Wymagania:
\begin{itemize}
	\item uprawnienia egzaminatora
	\item istnienie konkursu, który możesz edytować
\end{itemize}
Proces edycji istniejącego konkursu:
\begin{enumerate}
	\item Wejdź w zakładkę Contests na pasku nawigacji.
	\item Wyszukaj na liście konkursów konkurs, który możesz edytować (wcześniej utworzony przez ciebie). Naciśnij dostępny przy nim przycisk "Edit"
	\item Możesz edytować konkurs. Dostępne jest kilka opcji:
	\begin{itemize}
		\item zmiana nazwy konkursu
		\item dodanie nowego zadania
		\item zmiana statusu konkursu, z "nierozpoczętego" (Not started) na "w trakcie" (In progress), a później na "zakończony" (Completed)
	\end{itemize}
\end{enumerate}

\section{Dodawanie nowego zadania}
Wymagania:
\begin{itemize}
	\item uprawnienia egzaminatora
	\item istnienie konkursu, który możesz edytować
\end{itemize}
Proces dodawania nowego zadania:
\begin{enumerate}
	\item Wejdź w zakładkę Contests na pasku nawigacji.
	\item Wyszukaj na liście konkursów konkurs, który możesz edytować (wcześniej utworzony przez ciebie). Naciśnij dostępny przy nim przycisk "Edit".
	\item Możesz edytować konkurs. Po lewej stronie dostępne są pola związane z dodawaniem nowego zadania:
	\begin{itemize}
		\item nazwa zadania
		\item opis zadania
		\item polityka punktowania - możesz wybrać jedną z istniejących, lub utworzyć własną w zakładce Score policies
	\end{itemize}
\end{enumerate}

\section{Dodawanie nowej polityki punktowania}
Wymagania:
\begin{itemize}
	\item uprawnienia egzaminatora
\end{itemize}
Proces dodawania nowej polityki punktowania:
\begin{enumerate}
	\item Wejdź w zakładkę Score policies na pasku nawigacji.
	\item Możesz utworzyć nową politykę punktowania. Dostępne pola:
	\begin{itemize}
		\item nazwa
		\item opis
		\item zmiana ilości zasad punktowania
		\item pola zasad punktowania
	\end{itemize}
	\item Po utworzeniu, nowa polityka punktowania będzie dostępna przy dodawaniu zadania do konkursu
\end{enumerate}

\section{Dołączenie do konkursu}
Wymagania:
\begin{itemize}
	\item jesteś zalogowanym użytkownikiem
\end{itemize}
Obecnie dołączenie do konkursu zachodzi poprzez wysłanie rozwiązania na jedno z zadań w nim zawartych. Gdy to zrobisz, jesteś automatycznie rozpatrywany jako uczestnik konkursu.

\section{Rozwiązanie zadania}
Wymagania:
\begin{itemize}
	\item jesteś zalogowanym użytkownikiem
	\item istnieje przynajmniej jeden konkurs o statusie "w trakcie" (In progress)
\end{itemize}
Proces rozwiązania zadania:
\begin{enumerate}
	\item Wejdź w zakładkę Contests na pasku nawigacji.
	\item Wyszukaj na liście konkursów konkurs, którego zadanie chcesz rozwiązać (musi być w stanie In progress) i kliknij dostępny przy nim przycisk "View"
	\item Powinieneś zobaczyć teraz listę zadań dostępnych w wybranym konkursie. Kliknij przycisk "Solve" przy zadaniu, którego rozwiązanie chcesz przesłać.
	\item Teraz powinny być widoczne nazwa, opis zadania oraz przycisk służący do wyboru pliku. Wybierz plik zawierający rozwiązanie zadania z dysku i wciśnij przycisk "Send".
\end{enumerate}

\section{Wyświetlanie wyników konkursu przez jego twórcę}
Wymagania:
\begin{itemize}
	\item jesteś twórcą konkursu
	\item ten konkurs został zakończony
\end{itemize}
Proces wyświetlania wyników konkursu przez jego twórcę:
\begin{enumerate}
	\item Wejdź w zakładkę Contests na pasku nawigacji.
	\item Wyszukaj zakończony konkurs utworzony przez ciebie na liście konkursów. Wciśnij dostępny przy nim przycisk "Edit".
	\item Zamiast pól związanych z dodawaniem zadania, jest teraz dostępny przycisk "Results". Wciśnij go.
	\item Możesz zobaczyć wyniki konkursu. Jako jego twórca widzisz wyniki każdego uczestnika konkursu.
\end{enumerate}

\section{Wyświetlanie wyników konkursu przez jego uczestnika}
Wymagania:
\begin{itemize}
	\item jesteś zalogowanym użytkownikiem
	\item jesteś uczestnikiem zakończonego konkursu
\end{itemize}
Proces wyświetlania wyników konkursu przez jego uczestnika:
\begin{enumerate}
	\item Wejdź na stronę główną Algenic.
	\item Dostępna jest tutaj lista zakończonych konkursów, w których uczestniczyłeś. Wciśij przycisk "Results" przy interesującym cię konkursie.
	\item Możesz zobaczyć wyniki konkursu. Jako uczestnik widzisz tylko swoje własne wyniki zawierające m.in. zdobytą pozycję i sumę punktów.
\end{enumerate}
\end{document}